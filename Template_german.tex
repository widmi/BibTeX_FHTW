% Encoding = UTF-8

% LaTeX Vorlage zur Demonstration des BibTeX Styles.
%
% Erstellt von Michael Widrich im Jänner 2012, getestet durch W. Kubinger im März 2012 und angepasst im Jänner 2016 von Michael Widrich
%
% Ursprünglich erstellt für FH Technikum Wien, Studiengang Mechatronik/Robotik, neue Standards können abweichen.
%
% Disclaimer: I modified/created the files "HarvardFHTWMR.bst" and "literatur.bib" as a tool for anyone who wants to use it. I can not issue any guarantee about the correctness of its output. However, I hope it's able to save you time and trouble with the quotations.
%

\documentclass[a4paper,bibtotoc,oneside]{scrbook}

\usepackage{hyperref}

\usepackage[english, german]{babel}

\usepackage{amsmath}


\usepackage{harvard}


\begin{document}
%Festlegen des Zitier-Standards
\bibliographystyle{HarvardFHTWMR}%Zitierstandard FH Technikum Wien, Studiengang Mechatronik/Robotik
\citationstyle{dcu}%Correct citation-style (Harvardand, ";" between citations, "," between author and year)
\citationmode{abbr}%use "et al." with first citation
% Konfiguration der Makros
\newcommand{\citepic}[1]{(Quelle: \protect\cite{#1})}%Zitat: Bild
\newcommand{\citefig}[2]{(Quelle: \protect\cite{#1}, S. #2)}%Zitat: Bild aus Dokument
\newcommand{\citefigm}[2]{(Quelle: modifiziert "ubernommen aus \protect\cite{#1}, S. #2)}%Zitat: modifiziertes Bild aus Dokument
\newcommand{\citep}{\citeasnoun}%In-Line Zitiat entweder mit \citep{} oder \citeasnoun{}
\newcommand{\acessedthrough}{Verf{\"u}gbar unter:}%Für URL-Angabe
\newcommand{\acessedthroughp}{Verf{\"u}gbar bei:}%Für URL-Angabe (Geschützte Datenbank, Zugriff durch FH)
\newcommand{\acessedat}{Zugang am}%Für URL-Datum-Angabe
\newcommand{\singlepage}{S.}%Für Seitenangabe (einzelne Seite)
\newcommand{\multiplepages}{S.}%Für Seitenangabe (mehrere Seiten)
\newcommand{\chapternr}{K.}%Für Kapitelangabe
\renewcommand{\harvardand}{\&}%Harvardand in Zitaten
\newcommand{\abstractonly}{ausschließlich Abstract}
\newcommand{\edition}{. Auflage}%Angabe der Auflage


% title page:
\thispagestyle{empty}
\begin{center}
\begin{Huge}
HarvardFHTWMR BibTeX Style
\end{Huge}

\vspace{20pt}
\begin{LARGE}
Manual
\end{LARGE}
\end{center}


\newpage

\tableofcontents\thispagestyle{empty}
\newpage

\chapter{Anmerkungen zum MR-Zitierstandard}

\section{Verwendung von BibTeX}

Das Literaturverzeichnis wird automatisch generiert. Die Quellenangaben befinden sich in der Datei "`*.bib''. Zur einfachen Recherche der Literaturquellen-Daten empfiehlt sich auch \textit{google-scholar} (\url{scholar.google.de}). Dazu wird unter den "`Einstellungen'' (rechts oben) die Option  "`\textbf{Bibliographiemanager}: Links zum Importieren von Literaturverweisen in BibTeX anzeigen.'' aktiviert. Danach kann eine Literaturquelle gesucht und durch einen Klick auf "`In BibTeX importieren'' die Informationen per Copy\& Paste "ubernommen werden.\\

Kompiliert wird "uber bibtex mit pdflatex oder PS bzw. DVI.
F"ur diese Vorlage z.B. mit pdflatex$\rightarrow$bibtex$\rightarrow$bibtex$\rightarrow$pdflatex$\rightarrow$pdflatex. Die URL-Darstellung und die PDF Eigenschaften sowie die r"omische Nummerierung im PDF vor dem Hauptteil werden mit dem hypersetup bzw. \textbackslash pagenumbering\{\} generiert.\\


\section{Zitate ``im Text''}

In-Line Zitate k"onnen mit \textbackslash citeasnoun\{Quelle\} oder \textbackslash citep\{Quelle\} durchgef"uhrt werden. Bei mehreren Autoren sind die Quellen-Namen nach der Jahreszahl zu ordnen.

Beispiele f"ur Zitate ``im Text'': nach \citeasnoun{Braun07} und \citeasnoun{Kastner11} oder \citep{Kessler11}.\\

\section{Zitate ``nicht direkt im Text''}

``Normale'' Zitate in Klammer werden mit \textbackslash cite\{Quelle\} erzeugt. Bei mehreren Autoren sind die Quellen-Namen durch einen Beistrich zu trennen und nach der Jahreszahl zu ordnen.

Beispiele f"ur Zitate ``nicht direkt im Text'': \cite{Technikum11} bzw. \cite{Bach82,Zettler98,Astrom01}.

\section{Mehrere Werke des selben Autors im selben Jahr}

Bei mehreren Zitaten vom selben Autor im selben Jahr ist die Jahreszahl mit ``a'', ``b'', etc. zu erweitern (siehe *.bib-Datei).

Beispiele f"ur mehrere Zitate des selben Autors im selben Jahr: \cite{Aangerman09a} und \cite{Aangerman09b}.\\

\section{Abk"urzung mit ``et al.''}

Bei mehr als zwei Autoren wird automatisch mit ``et al.'' abgek"urzt: \cite{Zettler98}.

\section{Bildbeschriftungen}

Es gibt folgende Markos f"ur Quellenangabe bei fremden Fotos/Bildern:

\begin{itemize}
\item Referenz auf Foto/Bild: \textbackslash citepic\{Hemetsberger07\} $\overset{wird zu}{\Longrightarrow}$ \citepic{Hemetsberger07}
\item Referenz auf Foto/Bild aus Dokument: \textbackslash citefig\{Lund92\}\{99\} $\overset{wird zu}{\Longrightarrow}$ \citefig{Lund92}{99}
\item Referenz auf modifiziertes Foto/Bild aus Dokument: \textbackslash citefigm\{Lund92\}\{150\} $\overset{wird zu}{\Longrightarrow}$ \citefigm{Lund92}{150}
\end{itemize}\ \\

\section{Verf"ugbare Medientypen}
Folgende Medientypen sind zur Zeit m"oglich:

\begin{itemize}
\item Bachelorarbeit/Projektbericht/etc.: \cite{Baldinger10,Piringer11}.
\item Buch: \cite{Aangerman09a,Aangerman09b}.
\item Datenblatt, Leitfaden: \cite{Anglia10,Atmel11}.
\item E-Abstract (Nur Abstract verf"ugbar): \cite{Astrom01}.
\item E-Book: \cite{Kastner11}.
\item E-Book (Zugang durch FH): \cite{Kessler11}.
\item E-Magazin oder E-Journal: \cite{Lund92,Zinner07}.
\item E-Magazin oder E-Journal (Zugang durch FH): \cite{Bach82}.
\item Edited Book: \cite{Braun07,Braun10}.
\item Kapitel in einem Edited Book: \cite{Samson70,Smith75}.
\item Masterthese/Dissertation: \cite{Pohn10,Humenberger11}.
\item Normen: \cite{ISO98}.
\item Paper, Konferenzbeitrag, Journalartikel: \cite{Zettler98,Gesztesy00}.
\item Patent: \cite{Anderson10}.
\item Photographien, Bilder: \cite{Hemetsberger07}.
\item Photographien, Bilder (online): \cite{Dean08}.
\item Website: \cite{Technikum11}.
\item Zeitung: \cite{Slapper05}.
\item Zeitung (online): \cite{Chittenden03}.
\end{itemize}


% Literaturverzeichnis
\bibliography{literatur.bib}
\newpage

\end{document}
